\section{Conclusions}
We presented a software that performs an image analysis by feature extraction and classification from images containing seeds through a brand new unique, and easy-to-use framework. In detail, we propose two \emph{ImageJ} plugins, one capable of extracting morphological, textural and colour characteristics from images of seeds, and another one to classify the seeds into categories by using the extracted features. 
Moreover, we analysed and reported the performances of several categories of descriptors for seed images with four different classifiers, using an image database containing 3,386 samples of 120 plant species belonging to the \emph{Fabaceae} family. 
In general, some aspects can strongly influence both the feature extraction and the classification phases. Foremost, the quality of the original images to process can produce some artefacts in the segmentation phase. Secondly, the preprocessing step, such as the background cleaning, the spacing of the seeds during the acquisition, and the size of the seeds present in the images, need to be verified to consider only valid regions. Finally, the dataset represented a class imbalance problem.

The experiments carried out showed some interesting trends. The colour feature category alone produced the best results in every metric, either using kNN and Random Forest classifiers. However, apart from Naive Bayes and SVM results in which they were the best, the combination of all the three categories produced excellent results on average. Finally, the Random Forest was the only one to outrun 90\% both in metrics and in categories combination, showing its excellent versatility.

As a future direction, we plan to investigate Neural Networks features extraction and compare them with the traditional ones. Moreover, we would like to extend our approach to distinguish among seeds' genus and variety.

In conclusion, we realised two ImageJ plugins that non-expert operators can use to extract features from seed images and classify their classes accordingly. Finally, we compared several descriptors and classifiers to investigate the best categories and classification strategy, obtaining outstanding results in most cases.