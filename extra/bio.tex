\documentclass{letter}
\setlength{\textwidth}{6.5in}
\setlength{\textheight}{9in}
\setlength{\topmargin}{-0.5in}
\setlength{\oddsidemargin}{0pt}
\usepackage[pagebackref=true,breaklinks=true,letterpaper=true,colorlinks,bookmarks=false,linkcolor=blue,citecolor=blue]{hyperref}

\address{Andrea Loddo\\
	Dept. of Mathematics and Computer Science\\
	University of Cagliari,	09128, Cagliari, Italy \\
	Phone:(+39) 070 675 8503 \\
	Email: \texttt{andrea.loddo@unica.it}}
\signature{Andrea Loddo}

\include{macro}

\begin{document}
	\begin{letter}{}
		
		\opening{Dear Editors and Reviewers}
		
		We would like to submit the paper entitled ``An effective and friendly tool for seed image analysis'' after carefully following your excellent suggestions.
		%
		Below we give the authors biographies.
		%
		\begin{itemize}
		\item Dr. Andrea Loddo is a postdoc researcher at the Dept. of Mathematics and Computer Science, University of Cagliari. He earned his BSc  in 2012, MSc in 2014 and PhD in 2019, all from the University of Cagliari. His Ph.D thesis faced blood cells image analysis and classification issues, for creating automatic diagnosis tools as a support to medical analysis. He is author of 15 scientific manuscripts in peer-reviewed journals and international conference proceedings. His research interests include computer vision, biomedical image analysis, pattern recognition and machine learning. Currently, he is involved in a research project regarding human activity recognition. Moreover, he is pursuing a research activity for crypto scams analysis and biomedical image analysis for diagnosis support systems.
		\item Prof. Dr. Cecilia Di Ruberto received the MS in Computer Science from the University of Salerno, Italy, in 1990 and the Ph.D. degree in Computer Science from the University of Naples, Italy, in 1995. Currently she is Associate Professor of Computer Science at the Department of Mathematics and Computer Science, University of Cagliari, Italy. Her research interests include computer vision, image retrieval, medical image analysis, pattern recognition and machine learning.  She has been working in microscopic image analysis, in particular in blood smear image analysis for cell counting, malaria parasites detection and classification, and leukemia detection. She is author of more than 50 scientific papers in peer-reviewed journals and international conference proceedings.
		\item Dr. Mariano Ucchesu is a postdoctoral researcher at the Centro Conservazione Biodiversità (CCB), Dipartimento di Scienze della Vita e dell’Ambiente (DiSVA), Università degli Studi di Cagliari. He earned his PhD in 2014 from the University of Cagliari. Author of several scientific manuscripts in peer-reviewed journals and international conference proceedings related to this task, his research interests include paleobotanic, seeds analysis, plant science and archeobotanic.
		\item Dr. Prof. Gianluigi Bacchetta is the Coordinator of the Doctorate Course in Environmental and Applied Botany, Vice-coordinator of the Doctorate Course in Earth and Environmental Sciences and Technologies and Vice-Director of the Doctorate School in Engineering and Sciences for the Environment and Territory at the University of Cagliari. Also for the University of Cagliari he is scientific coordinator in the framework of the agreements stipulated with the Universidade de Evora (Evora, Portugal), Universisade do Rio Grande do Norte (Natal, Brasil), University of Tehran (Tehran, Iran), Universitat Jaume I (Castellon, España), Universidad de Castilla la Mancha (Toledo, España), Universitat de Valencia and Universitat Politècnica (Valencia, España).
		\item Dr. Prof.  Alessandra Mendes Pacheco Guerra Vale has a degree in Data Processing from the University Potiguar (1999), Master in Electrical Engineering from the Federal University of Rio Grande do Norte (2002) and PhD in Electrical and Computer Engineering from the Federal University of Rio Grande do Norte (2014). She is currently a professor at the Federal University of Rio Grande do Norte, Escola Agrícola de Jundiaí. She has experience in the area of Information Technology, working mainly on the following topics: intelligent systems, digital image processing, software analysis and development, object orientation, software development process, project management, computer supported cooperative work, database and telemedicine.
		\end{itemize}
		
		\closing{Yours sincerely,}
	\end{letter}
	
\end{document}

