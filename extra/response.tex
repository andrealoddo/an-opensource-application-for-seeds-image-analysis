\documentclass[]{article}

%opening
\title{Response to Comments}
\author{Loddo A., Di Ruberto C., \\Vale A.M.P.G.,	Ucchesu M.,	Soares J.M., Bacchetta G.}

\begin{document}
	
	\maketitle
	
	We thank the associate editor and reviewers for their constructive comments. 
	We have revised the manuscript according to the suggestions.
	For ease of review, we add a PDF file showing the differences between the original manuscript and the revised version. More precisely, text deleted from the old document (on the left side) and, hence, not in new version is highlighted red. Text added to the new document and, hence, not in the old one is highlighted green.
	We summarise the main changes below:
	\\
	\begin{enumerate}
		%
		\item We have completely reorganised the structure of the manuscript according to the guidelines of The Visual Computer. \\
		\item We have completely revised the English language and corrected typos. \\
		\item We fixed the issues in the bibliography, adapting them to the style of the journal. \\
		\item We have added a detailed comparison with deep learning methods. \\
		\item We have highlighted the main differences with our first presentation. \\
		%
	\end{enumerate}
	
	\section{Editor}
	\textbf{Whenever possible, please add relevant references from The Visual Computer. It is important to cite related work from the journal you are publishing in.} \\
	
	We have added several works from The Visual Computer, both in the Introduction and Related work sections, related to the task we have described in this manuscript.
	
	%%%% Rev 1
	\section{Reviewer 1}
	\begin{enumerate}
	\item \textbf{The literature review is not sufficient. more literature review is needed,} \\
    In the introductory section, we added more details about our approach and the context in which it fits. We have also created Sec.2 to expand on the state of the art, also highlighting traditional and deep learning approaches in plant science.
    
    \item \textbf{The writing and presentation of the paper have some issues. Please clear the paper.} \\
	We have completely revised the English language and corrected typos. In addition, we have corrected problems in the bibliography, adapting it to the style of the journal.
	
	\item \textbf{To give better insights to the reader other approaches can be added. Following references can be useful for the same.
	Linear discriminant multi-set canonical correlations analysis (LDMCCA): an efficient approach for feature fusion of finger biometrics.
	Saliency detection based on integrated features
	IoT Security Based on Iris Verification Using Multi-Algorithm Feature Level Fusion Scheme} \\
	
	We have added a significant section of related work to describe feature fusion approaches, citing those suggested as important examples of improved results.

    \item \textbf{The references are up to date but some are not in the right format.} \\
    We fixed the reference issues, adapting them to the style of the journal.

    \item \textbf{Conclusion section is poor, please improve.} 
    \item \textbf{The conclusions section should conclude that you have achieved from the study, contributions of the study to academics and practices. In addition, list the advantages and disadvantages of the proposed solution, as well as indicate the limitations of work. Further, mention the recommendations of future works.} \\
    We have updated the Conclusion section, giving more emphasis to the pros and cons of the proposed tool. We have included our own contributions in the Introduction and Conclusion sections.
    
    \item \textbf{Compare your results with the imp methods published in the literature}
    We have compared our approach with those of deep learning in section 5, and in particular in Table 6.
    
	\end{enumerate}
	
	\section{Reviewer 4}
	
	\begin{enumerate}
	\item \textbf{There are some layout problems in this paper. For example, the format of section title and text is same in Abstract, and the table title of Table 4 is under the table.} \\
	Although we have used the only Latex template available for this magazine, we have adapted it to the two-column format, which is closer to the final version once published. In any case, we have followed the official instructions at: https://www.springer.com/journal/371/submission-guidelines.
	
	\item \textbf{Some descriptions are redundant in this paper. For example, Introduction should be segmented for clarity.} \\
	We completely reformatted and reorganised the manuscript, carefully following the Visual Computer guidelines. In particular, we divided the previous Introduction section, in order to expand the introductory section, as well as the state-of-the-art work, adding more works and describing them in detail. 
	We have completely reformatted and reorganised the manuscript, carefully following Visual Computer guidelines. In particular, we have split the previous Introduction section, in order to expand it, as well as the state of the art, adding more works and describing them in detail. 
	The manuscript is finally organised with: (i) the Introduction section, which introduces the topic; (ii) the Related Work section, which presents the state of the art; (iii) the Materials and Methods section, in which we describe the datasets, the proposed ImageJ plugins, preprocessing, feature extraction and classification steps; (iv) the Results and Discussion section.

	\item \textbf{This paper lacks of visualization results in experimental results.}
	We have added two graphical results in section 5, namely Fig. 8 and 9 to show the results obtained with the different categories of features used.

	\item \textbf{And there is also lacking of comparison experiments with CNN-based methods to prove the effectiveness of the proposed tool.}
	We have compared our approach with those of deep learning in section 5, and in particular in Table 6.
	
	\item \textbf{For the references, please add more references published in recent years}
	We have created a Related Works section (Section 2) and added several updated works related to the field analysed in this article.
	\end{enumerate}
	
\end{document}