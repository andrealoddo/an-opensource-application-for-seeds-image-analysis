\section{Introduction}
Thanks to its wide range of applications, image analysis plays a vital role in the scientific field, mainly in quantitative measurements. Image visualisation and analysis methods are essential for understanding various medicinal characteristics \cite{Dirub_2015}, biology, haematology \cite{Dirub_2020}, botany and other biological branches in general. This is why biological image processing techniques have become more reliable with the development of fluorescence and high-resolution microscopes, with a profound impact on biological research giving the possibility to study the structural details of biological elements, such as organisms and parts thereof.
Biologists are increasingly interested in using image analysis techniques. ImageJ \cite{ImageJ} is defined as one of the standard image analysis software, as it is freely available, platform-independent and applicable by biological researchers to quantify laboratory tests. This paper presents a software for extracting features from biological organisms belonging to Carpology, the discipline that studies spermatophyte seeds and fruits from both a morphological and a structural perspective. This is of fundamental importance for Paleobotanica, Paleoenvironmental studies and ecology if applied to remains of the past (Paleocarpology).
Instead of manual analysis, the use of image analysis techniques on seeds has the following advantages: speed up the analysis process, minimise distortions created by natural light and microscopes, automatically identify specific characteristics based on image pixel values. The four main steps of an image analysis process are pre-processing, segmentation, features extraction, and classification \cite{Gonz_2018}.
Image pre-processing techniques prepare the image before analysing it to delete possible distortions or superfluous data or highlight and improve certain important features for further processing. The next step is segmentation. It subdivides the significant regions into sets of pixels with common features such as colour, intensity, or texture. The segmentation objective is to simplify and change image representation into something more significant and easier to analyse. Features extraction from regions of interest identified by segmentation is the subsequent step. The feature can be shape, texture or color based \cite{Dirub_2015}, \cite{Dirub_2009}. The final step is classification, i.e. the association of a label with the object under examination using supervised or unsupervised machine learning methods. 
The rest of the paper is organised as follows. The following section presents state of the art on seeds image analysis. Section 3 introduces the ImageJ environment used to implement our proposed plugin, described in Section 4. The experimental evaluation and the used dataset are discussed in Section 5, and, finally, in Section 6 we give the conclusions of the work.